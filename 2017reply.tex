\documentclass[10pt]{beamer}

\usetheme[progressbar=frametitle]{metropolis}
\usepackage{appendixnumberbeamer}

\usepackage{booktabs}
\usepackage{xspace}
\newcommand{\themename}{\textbf{\textsc{metropolis}}\xspace}

\usepackage{graphicx}
\usepackage{CTEX}
\setmainfont{Microsoft YaHei}
\setCJKmainfont{Microsoft YaHei}

\title{移动客户端开发通道面试陈述}
\subtitle{适用3级评审}
% \date{\today}
%\date{}
\author{张仁昌}
\institute{地图平台部/手图终端开发中心/地图引擎组}
% \titlegraphic{\hfill\includegraphics[height=1.5cm]{logo.pdf}}

\begin{document}

\maketitle

\begin{frame}{内容}
  \setbeamertemplate{section in toc}[sections numbered]
  \tableofcontents[hideallsubsections]
\end{frame}

\section{个人经历}

\begin{frame}[fragile]{个人经历}
	\begin{itemize}
		\item 2006.06 西北大学计算机系硕士毕业
		\item 2007.01-2014.12 灵图软件导航与车联网事业部副总经理
		\item 2014.12- 手图终端开发中心/地图引擎组
	\end{itemize}
\end{frame}

\begin{frame}[fragile]{主要工作成果}
	\begin{itemize}
		\item 特殊诱导系统
		\item iOS平台诱导引擎hotfix
	\end{itemize}
\end{frame}

\section{工作内容}

\begin{frame}{特殊诱导系统-挑战}
  \begin{itemize}
    \item 数据错误多,可控性差,如何追上竞品
    \item 复杂路口,程序分析能力有限
  \end{itemize}
\end{frame}

\begin{frame}{特殊诱导系统-修改前}
\includegraphics[width = 1.0\textwidth]{picture/before_specialguidance.pdf}
\end{frame}

\begin{frame}{特殊诱导系统-修改后}
\includegraphics[width = 1.0\textwidth]{picture/after_specialguidance.pdf}
\end{frame}

\begin{frame}{特殊诱导系统-流程图}
\includegraphics[width = 1.0\textwidth]{picture/specialguidance_flowchart.pdf}
\end{frame}

\begin{frame}{特殊诱导系统-对比数据}
  \begin{itemize}
    \item 与高德模型数据的对比
    \item sgn与gcn数量
  \end{itemize}
\end{frame}

\begin{frame}{特殊诱导系统-TODO List}
  \begin{itemize}
    \item 分析回流数据
    \item 诱导策略的验证
  \end{itemize}
\end{frame}

\begin{frame}{iOS热更新-问题}
  \begin{itemize}
    \item 苹果审核慢,重大问题不能及时更新
    \item 引擎大部分功能独立客户端,需要加快迭代速度
  \end{itemize}
\end{frame}

\begin{frame}{iOS热更新-方案选择}
  \begin{itemize}
    \item 使用JS重写引擎
    \item 策略全部可写入配置文件
	\item 开源编译器
	\end{itemize}
\end{frame}

\begin{frame}{iOS热更新-方案1优缺点}
  \begin{itemize}
    \item 使用JS重写引擎 
    \item 策略全部可写入配置文件,后台修改,灵活性差,复杂策略难以实现
	\item 开源编译器
	\end{itemize}
\end{frame}

\section{Elements}

\begin{frame}[fragile]{Typography}
      \begin{verbatim}The theme provides sensible defaults to
\emph{emphasize} text, \alert{accent} parts
or show \textbf{bold} results.\end{verbatim}

  \begin{center}becomes\end{center}

  The theme provides sensible defaults to \emph{emphasize} text,
  \alert{accent} parts or show \textbf{bold} results.
\end{frame}

\begin{frame}{Font feature test}
  \begin{itemize}
    \item Regular
    \item \textit{Italic}
    \item \textsc{SmallCaps}
    \item \textbf{Bold}
    \item \textbf{\textit{Bold Italic}}
    \item \textbf{\textsc{Bold SmallCaps}}
    \item \texttt{Monospace}
    \item \texttt{\textit{Monospace Italic}}
    \item \texttt{\textbf{Monospace Bold}}
    \item \texttt{\textbf{\textit{Monospace Bold Italic}}}
  \end{itemize}
\end{frame}

\begin{frame}{Lists}
  \begin{columns}[T,onlytextwidth]
    \column{0.33\textwidth}
      Items
      \begin{itemize}
        \item Milk \item Eggs \item Potatos
      \end{itemize}

    \column{0.33\textwidth}
      Enumerations
      \begin{enumerate}
        \item First, \item Second and \item Last.
      \end{enumerate}

    \column{0.33\textwidth}
      Descriptions
      \begin{description}
        \item[PowerPoint] Meeh. \item[Beamer] Yeeeha.
      \end{description}
  \end{columns}
\end{frame}
\begin{frame}{Animation}
  \begin{itemize}[<+- | alert@+>]
    \item \alert<4>{This is\only<4>{ really} important}
    \item Now this
    \item And now this
  \end{itemize}
\end{frame}
\begin{frame}{Tables}
  \begin{table}
    \caption{Largest cities in the world (source: Wikipedia)}
    \begin{tabular}{lr}
      \toprule
      City & Population\\
      \midrule
      Mexico City & 20,116,842\\
      Shanghai & 19,210,000\\
      Peking & 15,796,450\\
      Istanbul & 14,160,467\\
      \bottomrule
    \end{tabular}
  \end{table}
\end{frame}
\begin{frame}{Blocks}
  Three different block environments are pre-defined and may be styled with an
  optional background color.

  \begin{columns}[T,onlytextwidth]
    \column{0.5\textwidth}
      \begin{block}{Default}
        Block content.
      \end{block}

      \begin{alertblock}{Alert}
        Block content.
      \end{alertblock}

      \begin{exampleblock}{Example}
        Block content.
      \end{exampleblock}

    \column{0.5\textwidth}

      \metroset{block=fill}

      \begin{block}{Default}
        Block content.
      \end{block}

      \begin{alertblock}{Alert}
        Block content.
      \end{alertblock}

      \begin{exampleblock}{Example}
        Block content.
      \end{exampleblock}

  \end{columns}
\end{frame}
\begin{frame}{Math}
  \begin{equation*}
    e = \lim_{n\to \infty} \left(1 + \frac{1}{n}\right)^n
  \end{equation*}
\end{frame}
\begin{frame}{Quotes}
  \begin{quote}
    Veni, Vidi, Vici
  \end{quote}
\end{frame}

\section{Conclusion}

\begin{frame}{Summary}

  Get the source of this theme and the demo presentation from

  \begin{center}\url{github.com/matze/mtheme}\end{center}

  The theme \emph{itself} is licensed under a
  \href{http://creativecommons.org/licenses/by-sa/4.0/}{Creative Commons
  Attribution-ShareAlike 4.0 International License}.

  \begin{center}\ccbysa\end{center}

\end{frame}

{\setbeamercolor{palette primary}{fg=black, bg=yellow}
\begin{frame}[standout]
  Questions?
\end{frame}
}

\appendix

\begin{frame}[fragile]{Backup slides}
  Sometimes, it is useful to add slides at the end of your presentation to
  refer to during audience questions.

  The best way to do this is to include the \verb|appendixnumberbeamer|
  package in your preamble and call \verb|\appendix| before your backup slides.

  \themename will automatically turn off slide numbering and progress bars for
  slides in the appendix.
\end{frame}

\begin{frame}[allowframebreaks]{References}

  \bibliography{demo}
  \bibliographystyle{abbrv}

\end{frame}

\end{document}
