\documentclass[10pt]{beamer}

\usetheme[progressbar=frametitle]{metropolis}
\usepackage{appendixnumberbeamer}

\usepackage{booktabs}

\usepackage{xspace}
\newcommand{\themename}{\textbf{\textsc{metropolis}}\xspace}
\usepackage{tabularx}
\usepackage{graphicx}
\usepackage{subfigure}
%\usepackage{CTEX}
%\setmainfont{Microsoft YaHei}
%\setCJKmainfont{Microsoft YaHei}
\usepackage{array}
%\usepackage{math}
\newcolumntype{R}{>{\begin{math}}r<{\end{math}}}
	\usepackage{xeCJK}
\setCJKmainfont{MicrosoftYaHeiUILight}
\uselanguage{chinese}
\languagepath{chinese}

% 设置翻译
\deftranslation[to=chinese]{Contents}{目录}
\deftranslation[to=chinese]{Table}{\small 表}
\deftranslation[to=chinese]{Figure}{\footnotesize 图}
\deftranslation[to=chinese]{Theorem}{定理}
\deftranslation[to=chinese]{Corollary}{推论}
\deftranslation[to=chinese]{Definition}{定义}
\deftranslation[to=chinese]{Definitions}{定义}
\deftranslation[to=chinese]{Lemma}{引理}
\deftranslation[to=chinese]{Problem}{问题}
\deftranslation[to=chinese]{Solution}{解}
\deftranslation[to=chinese]{Fact}{事实}
\deftranslation[to=chinese]{Proof}{证明}

\title{移动客户端开发通道面试陈述}
% \subtitle{适用3级评审}
% \date{\today}
%\date{}
\author{张仁昌}
\institute{地图平台部/手图终端开发中心/地图引擎组}
% \titlegraphic{\hfill\includegraphics[height=1.5cm]{logo.pdf}}

\begin{document}

\maketitle

\begin{frame}{内容}
	\setbeamertemplate{section in toc}[sections numbered]
	\tableofcontents[hideallsubsections]
\end{frame}

\section{个人经历}

\begin{frame}[fragile]{个人经历}
	\begin{itemize}
		\item 2006.06 西北大学计算机系硕士毕业
		\item 2007.01-2014.12 灵图软件导航与车联网事业部副总经理
		\item 2014.12- 手图终端开发中心/地图引擎组
	\end{itemize}
\end{frame}

\begin{frame}[fragile]{近期主要工作成果}
	\begin{exampleblock}{}
		\begin{itemize}
			\item 特殊诱导系统
			\item iOS诱导引擎热更新
            \item 诱导引擎重构架构设计
            \item 王者荣耀诱导语音引擎
			\item 诱导引擎微信小程序移植
		\end{itemize}
	\end{exampleblock}
\end{frame}

\section{特殊诱导系统}
\begin{frame}{特殊诱导系统-背景}
	\begin{exampleblock}{用户看不懂地图}
		\begin{itemize}
			\item 手图转向标与实际路口不符
			\item 播报听不懂、不明确
		\end{itemize}
	\end{exampleblock}
	\begin{exampleblock}{问题原因}
		\begin{itemize}
			\item 数据制作错误
			\item 路口太复杂,算法分析能力有限
			\item 数据、算法都没有错,人的感知不一样
		\end{itemize}
	\end{exampleblock}
\end{frame}

\begin{frame}{特殊诱导系统badcase-数据问题}
	\begin{figure}
		\begin{minipage}[htbp]{0.45\textwidth}
			\centering
				\includegraphics[width = 0.45\columnwidth, height=0.45\columnwidth]{picture/data_3to1.png}
				\includegraphics[width = 0.45\columnwidth, height = 0.45\columnwidth]{picture/data_3to1_para.png}
				\caption{数据错误}
		\end{minipage}
					\begin{minipage}[htbp]{0.45\textwidth}
						\centering
						\includegraphics[width = 0.45\columnwidth, height = 0.45\columnwidth]{picture/data_3to21.png}
						\includegraphics[width = 0.45\columnwidth, height = 0.45\columnwidth]{picture/data_3to21_para.png}
						\caption{右转专用道属性做错}
					\end{minipage}

					\begin{minipage}[htbp]{0.45\textwidth}
						\centering
						\includegraphics[width = 0.45\columnwidth, height = 0.45\columnwidth]{picture/data_2360_6to81.png}
						\includegraphics[width = 0.45\columnwidth, height = 0.45\columnwidth]{picture/data_2360_6to81_para.png}
						\caption{入边做成辅路}
					\end{minipage}	
					\begin{minipage}[htbp]{0.45\textwidth}
						\centering
						\includegraphics[width = 0.45\columnwidth, height = 0.45\columnwidth]{picture/data_3102_8to1.png}
						\includegraphics[width = 0.45\columnwidth, height = 0.45\columnwidth]{picture/data_3102_8to1_para.png}
						\caption{内link做成普通路}
					\end{minipage}	
	\end{figure}	
\end{frame}

\begin{frame}{特殊诱导系统badcase-策略问题}
	\begin{figure}
		\begin{minipage}[htbp]{0.45\textwidth}
			\centering
				\includegraphics[width = 0.5\columnwidth, height=0.5\columnwidth]{picture/program_22to3.png}
				\includegraphics[width = 0.5\columnwidth, height = 0.5\columnwidth]{picture/program_22to3_para.png}
				\caption{延伸失败}
		\end{minipage}
					\begin{minipage}[htbp]{0.45\textwidth}
						\centering
						\includegraphics[width = 0.5\columnwidth, height = 0.5\columnwidth]{picture/program_211_3to23.png}
						\includegraphics[width = 0.5\columnwidth, height = 0.5\columnwidth]{picture/program_22to3_para.png}
						\caption{复杂难以分析}
					\end{minipage}

					\begin{minipage}[htbp]{0.45\textwidth}
						\centering
						\includegraphics[width = 0.5\columnwidth, height = 0.5\columnwidth]{picture/program_2695_7to8.png}
						\includegraphics[width = 0.5\columnwidth, height = 0.5\columnwidth]{picture/program_2695_7to8_para.png}
						\caption{难以分析}
					\end{minipage}	
					\begin{minipage}[htbp]{0.45\textwidth}
						\centering
						\includegraphics[width = 0.5\columnwidth, height = 0.5\columnwidth]{picture/program_2787_miss_turnround.png}
						\includegraphics[width = 0.5\columnwidth, height = 0.5\columnwidth]{picture/program_2787_miss_turnround_para.png}
						\caption{掉头路太长分析错误}
					\end{minipage}	
	\end{figure}	
\end{frame}

\begin{frame}{特殊诱导系统-目标}
	\begin{exampleblock}{需要设计一个系统,完美解决以下问题}
		\begin{itemize}
			\item 目标:数据或策略错误的位置,必须纠正
			\item 目标:不符合人的行为的位置,尽量纠正使人理解
			\item 目标:如果数据或引擎出错,具备快速更新的能力
			\item 目标:修改策略后效果的验证
			\item 目标:变被动接收问题为主动收集问题
		\end{itemize}
	\end{exampleblock}
	\begin{exampleblock{核心思想}
		\begin{itemize}
			\item 特殊诱导中间层数据
			\item 诱导策略的发现,修改,验证,回流的闭环,需要修改图
		\end{itemize}
	\end{exampleblock}
%	\begin{exampleblock}{应对}
%		\begin{itemize}
%			\item 增加中间层数据
%			\item 快速更新引擎的能力
%		\end{itemize}
%	\end{exampleblock}
\end{frame}

%\begin{frame}{特殊诱导系统-挑战及应对}
%	\begin{exampleblock}{挑战}
%		\begin{itemize}
%			\item 数据错误多,可控性差,如何追上竞品
%			\item 特殊路口,程序分析能力有限
%		\end{itemize}
%	\end{exampleblock}
%	\begin{exampleblock}{应对}
%		\begin{itemize}
%			\item 增加中间层数据
%			\item 快速更新引擎的能力
%		\end{itemize}
%	\end{exampleblock}
%\end{frame}

\begin{frame}{特殊诱导系统-原有结构图}
	\includegraphics[width = 1.0\textwidth]{picture/before_specialguidance.png}
\end{frame}

\begin{frame}{特殊诱导系统-期望结构图}
	\includegraphics[width = 1.0\textwidth]{picture/after_specialguidance.png}
\end{frame}

\begin{frame}{特殊诱导系统-挑战及应对}
	\alert{挑战-特殊诱导数据如何存储方式?}
	\begin{exampleblock}{解决方案}
		\begin{itemize}
			\item 使用postgres数据库作为展示,操作数据
			\item 使用mif数据作为编译数据
			\item 编写导出脚本保证数据一致性-流程图画错要修改
		\end{itemize}
	\end{exampleblock}
	\alert{挑战-如何在早期验证方案可行性?}
	\begin{exampleblock}{解决方案}
		\begin{itemize}
			\item 利用离线导航模式,仅用一个月时间完成了离线编译器,诱导引擎的工作,实际路测OK
			\item 利用二次索引的方式保证不改变数据块头的结构,保证了兼容性
		\end{itemize}
	\end{exampleblock}
		\alert{挑战-如何验证新改策略的效果?}
	\begin{exampleblock}{解决方案}
		\begin{itemize}
			\item 新增goodcase,badcase库
			\item 增加回流机制发现用户问题
		\end{itemize}
	\end{exampleblock}
%	\begin{exampleblock}{应对}
%		\begin{itemize}
%			\item 增加中间层数据
%			\item 快速更新引擎的能力
%		\end{itemize}
%	\end{exampleblock}
\end{frame}

\begin{frame}{特殊诱导系统-流程图}
	\includegraphics[width = 1.0\textwidth]{picture/specialguidance_flowchart.png}
\end{frame}

\begin{frame}{特殊诱导系统-与高德对比数据}
	\begin{table}
		\caption{各版本路口转向对比}
					\begin{tabularx}{0.9\textwidth}{X|X|X|X|X}
						\toprule 
						\footnotesize 版本 & \footnotesize 相同路口数 & \footnotesize 缺失路口数 & \footnotesize 不同路口数 & \footnotesize 多出路口数\\
						\midrule
						6.1 & 57.00\% & 11.21\% & 7.02\% & 24.00\%\\
						6.2 & 61.70\% & 12.74\% & 5.35\% & 20.21\%\\
						6.3 & 74.73\% & 11.68\% & 5.53\% & 8.06\%\\
						6.4 & 75.83\% & 10.72\% & 5.69\% & 7.76\%\\
						6.5 & 78.35\% & 9.93\% & 5.25\% & 6.47\%\\
						6.6 & 78.86\% & 9.55\% & 4.68\% & 6.42\%\\
						6.7 & 79.27\% & 9.73\% & 4.28\% & 6.22\%\\
						6.8 & 78.87\% & 10.15\% & 4.52\% & 6.11\%\\
						6.9 & 79.75\% & 10.03\% & 4.61\% & 5.30\%\\
						\bottomrule
					\end{tabularx}
	\end{table}      
\end{frame}

\begin{frame}{特殊诱导系统-差异路口对比数据}
	\begin{columns}[T,onlytextwidth]
		\column{0.45\textwidth}
		\begin{table}
			\caption{特殊诱导数据}
							\begin{tabularx}{\textwidth}{XX}
								\toprule
									类型 & 数量\\
									\midrule
									%select count(*) from specialguidance where type=1 and actionid is NULL
									特殊转向 & 3783\\
									%select count(*) from specialguidance where type=1 and actionid is not NULL
									特殊播报 & 90\\
									%select count(*) from specialguidance where state = 3
									车道线 & 5\\
									%select count(*) from specialguidance where state = 3
									特殊单点 & 15\\
									\bottomrule
							\end{tabularx}
		\end{table}		
		\column{0.45\textwidth}
		\begin{table}
			\caption{验证库数据}
							\begin{tabularx}{\textwidth}{XX}
								\toprule
									类型 & 数量\\
									\midrule
									%select count(*) from tracks where state = 1
									good case & 41896\\
									bad case & 1019\\
									invalid & 369\\
									未检测 & 19403\\
									\bottomrule
							\end{tabularx}
		\end{table}		
	\end{columns}
	\begin{columns}[T,onlytextwidth]
		\column{0.45\textwidth}
		\begin{table}
			\caption{人工采样校验}
					\begin{tabularx}{1\textwidth}{XXX}
						\toprule
						版本 & 腾讯好 & 高德好\\
						\midrule
						6.6 & 5.70\% & 3.02\%\\
						6.7 & 79.27\% & 9.73\%\\
						\bottomrule
					\end{tabularx}
		\end{table}
		\column{0.45\textwidth}
			\begin{table}
				\caption{实际路测}
					\begin{tabularx}{1\textwidth}{XXX}
						\toprule
						城市 & 腾讯准 & 高德准\\
						\midrule
						上海 & 86.44\% & 86.44\%\\
						深圳 & 85.77\% & 77.88\%\\
						\bottomrule
					\end{tabularx}
			\end{table}
	\end{columns}
\end{frame}

\begin{frame}{特殊诱导系统-总结}
	\begin{exampleblock}{总结}
		\begin{itemize}
			\item 建立了诱导策略的发现,回流,修复的闭环
			\item 模型准确率已经超过高德
		\end{itemize}
	\end{exampleblock}
	\begin{exampleblock}{todo}
		\begin{itemize}
			\item 诱导策略的追赶
			\item 参考点,电子眼,车道线策略的逐步更新
		\end{itemize}
	\end{exampleblock}
\end{frame}

\section{诱导引擎热更新}

\begin{frame}{iOS热更新-问题}
	\begin{itemize}
		\item 苹果审核慢,重大问题不能及时更新
		\item 引擎大部分功能独立客户端,需要加快迭代速度
	\end{itemize}
\end{frame}

\begin{frame}{iOS热更新-方案选择}
	\begin{itemize}
		\item 使用JS重写引擎
		\item 策略全部写入配置文件
		\item 编译器翻译到JS
	\end{itemize}
\end{frame}

\begin{frame}{iOS热更新-方案1优缺点}
	\begin{exampleblock}{优点}
		\begin{itemize}
			\item 可控性强
			\item 性能优化空间大
			\item 后台服务不需要修改
		\end{itemize}
	\end{exampleblock}
	\begin{exampleblock}{缺点}
		\begin{itemize}
			\item 需要人力
			\item 需要保证两个版本统一
			\item 开发时间长
		\end{itemize}
	\end{exampleblock}
\end{frame}

\begin{frame}{iOS热更新-方案2优缺点}
	\begin{exampleblock}{优点}
		\begin{itemize}
			\item 技术可控
			\item 无需新语言
		\end{itemize}
	\end{exampleblock}
	\begin{exampleblock}{缺点}
		\begin{itemize}
			\item 需要后台服务配合修改
			\item 配置文件规则复杂
			\item 灵活性差
			\item 复杂策略难以实现
		\end{itemize}
	\end{exampleblock}
\end{frame}

\begin{frame}{iOS热更新-方案3优缺点}
	\begin{exampleblock}{优点}
		\begin{itemize}
			\item 后台服务不需要修改
			\item 双版本内容一致
			\item 无需额外开发人员
		\end{itemize}
	\end{exampleblock}
	\begin{exampleblock}{缺点}
		\begin{itemize}
			\item C与JS交互部分数据传递
			\item 回调函数处理
			\item 可控性较低
		\end{itemize}
	\end{exampleblock}
\end{frame}

\begin{frame}{热更新-类图}
	\includegraphics[width = 1.0\textwidth]{picture/hotfix_class_diagram.png}
\end{frame}

\begin{frame}{热更新-流程图}
	\includegraphics[width = 1.0\textwidth]{picture/hotfix_flowchart.png}
\end{frame}

\section{总结}

\begin{frame}{专业领域专长}
	\alert{专业领域专长}
	\begin{itemize}
		\item 熟悉地图各个模块的内容(搜索,地图,算路,诱导)
		\item 擅长跨平台开发
		\item 独自完成基于卡尔曼滤波的误偏航优化项目,误偏航次数减少53\%
		\item 站在更高的角度看问题,对一个问题习惯放在一个系统中考虑
	\end{itemize}
\end{frame}

\begin{frame}{专业影响力和贡献}
	\alert{专业影响力和贡献}
	\begin{itemize}
		\item 跟产品同学分享自己多年的导航经验,主导了腾讯地图诱导引擎的开发
		\item 指导组内员工完成导航引擎重构,实现部分诱导策略可配 
		\item 担任过一次新员工导师
		\item 对诱导数据回流项目进行设计上的修正
	\end{itemize}
	\alert{KM文章}
	\begin{itemize}
		\item http://km.oa.com/articles/show/293334
		\item http://km.oa.com/articles/show/281996 
		\item http://km.oa.com/articles/show/291038
	\end{itemize}	
\end{frame}

\begin{frame}{感想}
	\begin{itemize}
		\item 地图行业不论发展到什么地步,最核心竞争力始终是基础能力,即定位、查询、规划、导航的综合能力。
		\item 细节决定成败。
	\end{itemize}
\end{frame}

\begin{frame}{}
	谢谢
\end{frame}

\end{document}
